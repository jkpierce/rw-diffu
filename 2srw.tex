\documentclass[11pt]{article}
% General document formatting
\usepackage[margin=0.75in]{geometry}
\usepackage[parfill]{parskip}
\usepackage[utf8]{inputenc}
\usepackage{subfig}         % side-by-side figures 
% Related to math
\usepackage{amsmath,amssymb,amsfonts,amsthm}
\usepackage{graphicx}
\usepackage{natbib}
\usepackage{titling}
\usepackage{hyperref}
\usepackage{wrapfig}
\usepackage{booktabs} % for wrapping tabulars in accord with
\bibliographystyle{agu}
\setlength{\droptitle}{-5em}   % This is your set screw

%\usepackage[math]{kurier}
\newcommand\be{\begin{equation}} % shortcut to start eq envs 
\newcommand\ee{\end{equation}}   % shortcut to end eq envs
\newcommand\ol{\overline}        % shortcut to draw overline 
\newcommand\bra{\langle}
\newcommand\ket{\rangle}
\newcommand\El{\mathcal{L}}
\newcommand\tg{\tilde{g}}
\newcommand\tG{\tilde{G}}
\begin{document}

\title{Analytical derivation of three-stage bedload diffusion}
\author{Kevin Pierce}
\maketitle

\section{Two-state random walk formalism}
The formalism of two-state random walks \citep[e.g.][]{Weiss1976,Weiss1994,Masoliver2016,Masoliver2017} provides new traction on the problem of 3-stage diffusion in bedload transport \citep[e.g.][]{Nikora2001a,Nikora2002,Zhang2012,Fan2016}.
Assuming residence time distributions $\psi_\pm(t)$ in each state with associated survival probabilities $\Psi_\pm(t) = \int_t^\infty dt' \psi_\pm(t')$, probabilities of moving a distance $x$ up to time $t$ within a sojourn $f_\pm(x,t)$, denoting $g_\pm(x,t) =f_\pm(x,t)\psi_\pm(t)$ and $G_\pm(x,t) = f_\pm(x,t)\Psi_\pm(x,t)$, and letting $\omega_\pm(x,t)$ be the probability that a sojourn in state $\pm$ ends at $x,t$ provides
\be  p_\pm(x,t) = \theta_+ G_\pm(x,t) + \int_0^\infty dx' \int_0^\infty dt' \omega_{\mp}(x',t')G_\pm(x-x',t-t')\ee
and 
\be \omega_\pm(x,t) = \theta_\pm g_\pm(x,t) + \int_0^\infty dx' \int_0^\infty dt' \omega_\mp(x',t')g_\pm(x-x',t-t').\label{eq:two}\ee
Here $\theta_\pm$ are the initial probabilities of being in each state with $\theta_+ + \theta_-=1.$
The probability of the two-state random walker being at position $x$ at time $t$ is 
\be p(x,t) = p_-(x,t)+p_+(x,t).\ee
Denoting the laplace transform with respect to the variable $q$ as $\El_q$ and associating variables $\eta$ and $s$ with $\El_x$ and $\El_t$, 
taking $\El_x \El_t$ of (\ref{eq:two}) provides a much simpler algebraic problem for the probability $p$:

\be
\tilde{\omega}_\pm = 
\frac{[\theta_\pm+ \theta_\mp \tilde{g}_\mp]\tilde{g}_\pm}{1-\tilde{g}_-\tilde{g}_+}
\ee
and (c.f. \citet{Masoliver2016} eq. 20)
\be
\tilde{p}_\pm = \Big( \theta_\pm + \tilde{\omega}_\mp \Big)\tilde{G}_\pm = \frac{\theta_\pm + \theta_\mp \tilde{g}_\mp}{1- \tilde{g}_-\tilde{g}_+}\tilde{G}_\pm.
\ee
Therefore the double transform of the joint PDF reads (c.f. \citet{Masoliver2016} eq. )
\be
\tilde{p}(\eta,s) = \frac{\big\{ \theta_- + \theta_+ \tg_+\big\}\tG_- + \big\{\theta_+ + \theta_-\tg_-\big\}\tG_+}{1-\tg_-\tg_+}
\label{eq:mw}\ee
This is a direct generalization of the famous Montroll-Weiss formula for a single state continuous-time random walk \citep[e.g.][]{Weiss1994}.

For the evaluation of this formula, a useful fact to take account of is 
\be \El_t\{ \Psi_\pm(t);s\} = \int_0^\infty dt e^{-st} \int_t^\infty dt' \psi_\pm(t') = \frac{1-\tilde{\psi}(s)}{s}. \ee
Finally, owing to the definition of the double Laplace transform of $p(x,t)$:
\be \tilde{p}(\eta,s) = \int_0^\infty dt e^{-st} \int_0^\infty dx e^{-\eta x} p(x,t)\ee
we see the (double) inverse transform $p(x,t)$ may not be necessary to study the scaling of the variance $\bra x(t)^2 \ket = \int_0^\infty x^2 p(x,t)$ of the bedload tracer cloud since its Laplace transform follows from the second derivative of the double transform of $p$ wrt $\eta$:
\be \El_t\{ \bra x(t)^2 \ket;s\} = \partial_\eta^2 \tilde{p}(\eta,s)\Big|_{\eta=0}. \label{eq:var}
\ee
I hope this allows evaluation of three-stage diffusion.
More generally, 
\be \El_t\{\bra x(t)^k\ket;s\} = (-)^k\partial_\eta^k \tilde{p}(\eta,s)\Big|_{\eta=0}.\ee

Actually, it may be easier to work with a cumulant-type expression since unlike \citet{Masoliver2016,Masoliver2017} we consider asymmetric processes where $\bra x(t) \ket \neq 0$ necessarily.

\section{The Einstein theory}
Taking $g_-(x,t) = \delta(x)k_- e^{-k_-t}$ (rest) and $g_+(x,t) = k_+ e^{-k_+ x}\delta(t) $ (step) reproduces the \citet{Einstein1937} diffusion theory.
In this case the double transforms are: 
\be \tilde{g}_-(\eta,s) = \frac{k_-}{k_-+s}\ee
\be \tilde{g}_+(\eta,s) = \frac{k_+}{k_++\eta}, \ee
and the survival functions are 
\be \Psi_-(t) = \int_t^\infty dt' k_-e^{-k_- t'} = e^{-k_- t} \ee
and 
\be \Psi_+(t) = \int_t^\infty dt' \delta(t') = 0 ,  \ee
meaning $G_-(x,t) = \delta(x) e^{-k_- t}$ and $G_+(x,t) = 0,$ providing
\be \tG_-(\eta,s) = \frac{1}{k_-+s}.\ee
Taking $\theta_-=1$ and $\theta_+=0$, so the dynamics start at rest, the MW generalization (\ref{eq:mw}) is
\be \tilde{p}(\eta, s) = \frac{\tG_-}{1-\tg_-\tg_+} = \frac{1}{s + \frac{k_- \eta}{k_+ + \eta }}.\label{eq:mwgen}\ee
The Laplace transform of the mean is 
\be \El_t\{\bra x(t) \ket;s\} = \frac{k_-}{s^2 k_+}
\ee
so in real space it's $\bra x(t) \ket = k_- t/k_+$ as expected \citep[e.g.][]{Einstein1937, Nakagawa1976}.
The Laplace transform of the second moment is 
\be \El \{\bra x(t)^2 \ket; s\} = 2\Big(\frac{k_-}{k_+}\Big)^2 \Big[ \frac{1}{k_-}\frac{1}{s^2} + \frac{1}{s^3}\Big], \ee
so using the inversion formula 
\be \El\Big\{\frac{1}{s^{k+1}};s\Big\} = \frac{t^k}{k!}\ee
implies a second moment $\bra x(t)^2 \ket = (k_-/k_+)^2[2t/k_-+t^2] $
and a variance exemplifying the normal diffusion of bedload:
\be \sigma_x^2 = \text{var}\{x(t)\} = \frac{2k_-}{k_+^2}t.\ee
The diffusivity $D$ is given by the square of the mean step distance $1/k_+$ divided by the mean resting time $1/k_-$:
\be D_\text{Einstein} = \frac{k_-}{k_+^2}.\ee

Of course, for the Einstein theory a closed form solution of the pdf $p(x,t)$ is possible to obtain \citep[e.g.][]{Einstein1937, Hubbell1964, Daly2006,Daly2019}.
The first transform in (\ref{eq:mwgen}) inverts easily for \citep[i.e.][1.1.1.2]{Prudnikov1986}
\be \tilde{p}(\eta,t) = \exp\Big\{-\frac{k_-\eta }{k_+ + \eta} t\Big\}.\ee
Incidentally this single Laplace transform of $p$ provides the cumulant generating function $c(\eta,t) = \log \tilde{p}(-\eta,t)$ from which the variance follows from a more simple computation: $\sigma_x^2(t) = \partial_\eta^2 c(\eta,t)\big|_{\eta=0}.$
The second inversion follows from \citet[][2.2.2.8]{Prudnikov1986} noting $\El_x^{-1}\{1\} = \delta(x).$ The result is \citep[e.g.][]{Daly2019}
\be p(x,t) = e^{-k_- t -k_+ x}\Big\{ \sqrt{\frac{k_- k_+ t}{ x}} I_1\Big( 2\sqrt{k_-k_+ x t}\Big) + \delta(x) \Big\} .\ee
This exact solution has been the benchmark theory of bedload diffusion for over 100 years. I have only derived it within a more general framework of multi-state random walks \citep[e.g.][]{Weiss1994}.

\section{The Lisle Theory}
Apart from formulations of \citet{Einstein1937} using different step length and resting time distributions than exponential \citep[e.g.][]{Sayre1965}, the first significant advancement from \citet{Einstein1937} was due to \citet{Lisle1998}. 
I need to carefully investigate whether \citet{Gordon1972} did it too.
They imparted a finite duration to bedload motions, rather than considering them instantaneous as Einstein did. In this way, they derived two stages of bedload diffusion.
This approach is closely related to the so-called persistent diffusion model \citep{Balakrishnan1988,VanDenBroeck1990}, the diffusion of a particle driven by dichotomous Markov noise \citep[e.g.][]{Horsthemke1984,Risken1989,Bena2006}. 
The mathematics were essentially developed by Takacs (1957).

It is obtained by the choice $g_-(x,t) = \delta(x) k_- e^{-k_- t}$ (rest) and $g_+(x,t) = \delta(x-vt)k_+ e^{-k_+ t} $ (motion).
Hence motions occur with velocity $v$ for a duration characterized by an exponential distribution with mean $1/k_+$, while rests occur for a duration characterized by an exponential distribution with mean $1/k_-$. For convenience we again assume that all particles start at rest, meaning $\theta_- = 1$ and $\theta_+=0$.

In this case the Laplace transforms are
\be \tg_-(\eta,s) = \frac{k_-}{k_- + s}, \ee 
\be \tg_+(\eta,s) = \frac{k_+}{k_+ + \eta v + s }\ee

and
\be \tG_\pm(\eta,s) = \frac{1}{k_\pm}\tg_\pm(\eta,s).\ee

Plugging these into (\ref{eq:mw}) gives
\be \tilde{p}(\eta,s) = \frac{k + s + \eta v }{v(k_-+s) \eta  + k s + s^2 },\ee
where $k = k_- + k_+$. Following \citet{Lisle1998}it seems the best way to go is to invert first over $\eta$.
This inversion follows using the identities
\be \El\Big\{\frac{a}{bs+c};s\Big\} = \frac{a}{b}e^{-cx/b}\ee
and
\be \El\Big\{\frac{as}{bs+c};s\Big\} = \frac{a}{b}\Big[\delta(x) - \frac{c}{b}e^{-cx/b}\Big],\ee
and it results in 
\be \tilde{p}(x,s) = \frac{k(k+s)}{v(k_-+s)^2} \exp\Big[-\frac{s(k+s)}{v(k_-+s)}x\Big] + \frac{1}{k_-+s}\delta(x) .\ee
There's a need for strong transform calculus knowledge to proceed.
One necessary property involves the equivalence of shifting $x$ in real space and multiplying by an exponential factor in transform space: 
\be \El\{f(x+a);s\} = e^{as}\El\{f(x);s\},\ee
which means $\El^{-1} \{e^{as}\tilde{f}(s);x\} = \El^{-1}\{\tilde{f}(s);x+a\}$.
Then you need 
\be \El^{-1}\big\{\frac{1}{s^\nu}\exp(-a/s);t\} = \Big(\frac{t}{a}\Big)^{(\nu-1)/2}I_{\nu-1}\big(2\sqrt{a t}\big) \ee
from \citet[][2.2.2.1]{Prudnikov1986} to finally obtain 
\begin{multline}
p(x,t) = \delta(x) e^{-k_- t} + \frac{k}{v}\exp\Big[-\frac{k_+x }{v} - k_-\Big(t-\frac{x}{v}\Big)\Big]\Theta(t-x/v)\\ 
 \times \Big\{ I_0\Big(2\sqrt{\frac{k_-k_+x}{v}\big(t-\frac{x}{v}\big)}\Big) \\
\sqrt{\frac{k_+ v(t-x/v)}{k_-x}}I_1\Big(2\sqrt{\frac{k_-k_+x}{v}\big(t-\frac{x}{v}\big)}\Big) 
\Big\},
\end{multline}
a non-trivial result. A similar result was obtained by \citet{Lisle1998} in context of rain splash transport and it was identified as a generalization of \citet{Einstein1937}. Hopefully, the only difference between these two formulations (mine and Lisle's) is that I started at rest $\theta_0=1$ while they started in motion $\theta_1=1$.
Interestingly, this result has hardly been followed up. 
I'll come back to this and study it more deeply later. 
For future reference, the non-dimensionalization made by Lisle is $\xi = k_+ x /v$ and $\tau = k_-(t-x/v)$ in your notation.
In this notation the result appears as (for $x\neq0$)
\be p(x,t) = \frac{k}{v}e^{-\xi-\tau}\Theta(\tau)\Theta(\xi)\Big\{I_0\big(2\sqrt{\xi \tau}\big) +\frac{k_+}{k_-}\sqrt{\frac{\tau}{\xi}}I_1\big(2\sqrt{\xi\tau}\big) \Big\}\ee
which is close enough to \citet{Lisle1998} to make me reasonably comfortable the only difference is the initial condition.
Of course, the original intention is the variance of the bedload tracers.
The easiest crank to turn is to take two derivatives wrt $\eta$ of the double transform $\tilde{p}(\eta,s).$

The first derivative gives 
\be \El\{\bra x(t) \ket;s\} = vk_- \frac{1}{s^2(s+k)},\ee
while the second gives 
\be \El\{\bra x^2(t) \ket ; s\} = 2 v^2 k_- \frac{s+k_-}{s^3(s+k)^2}. \ee
The necessary Laplace transforms to evaluate the mean and variance are \citep[][2.1.2.33]{Prudnikov1986}:
\be \El^{-1}\Big\{p^{-2}(s+a)^{-1};x\Big\} = \frac{1}{a^2}(e^{-ax}+ax-1),\ee
\citep[][2.1.2.49]{Prudnikov1986}
\be \El^{-1}\Big\{p^{-2}(s+a)^{-2};x\Big\} = \frac{1}{a^3}[ax - 2 + (ax+2)e^{-ax}],\ee
and 
\be \El^{-1}\Big\{p^{-3}(s+a)^{-2};x\Big\} = \frac{1}{a^4}\Big[\frac{(ax)^2}{2} - 2ax - (ax+3)e^{-ax} + 3\Big] \ee
obtained using the previous formula with the differentiation formula to account for the additional factor of $1/p$.
Using these results, the mean is 
\be \bra x(t) \ket = \frac{vk_-}{k^2}(kt -1 + e^{-kt}).\ee
\bibliography{biblio}
\end{document}



