\documentclass[11pt]{article}
% General document formatting
\usepackage[margin=0.75in]{geometry}
\usepackage[parfill]{parskip}
\usepackage[utf8]{inputenc}
\usepackage{subfig}         % side-by-side figures 
% Related to math
\usepackage{amsmath,amssymb,amsfonts,amsthm}
\usepackage{graphicx}
\usepackage{natbib}
\usepackage{titling}
\usepackage{hyperref}
\usepackage{wrapfig}
\usepackage{booktabs} % for wrapping tabulars in accord with
\bibliographystyle{agu}
\setlength{\droptitle}{-5em}   % This is your set screw

%\usepackage[math]{kurier}
\newcommand\be{\begin{equation}} % shortcut to start eq envs 
\newcommand\ee{\end{equation}}   % shortcut to end eq envs
\newcommand\ol{\overline}        % shortcut to draw overline 
\newcommand\bra{\langle}
\newcommand\ket{\rangle}
\newcommand\El{\mathcal{L}}
\newcommand\tg{\tilde{g}}
\newcommand\tG{\tilde{G}}
\newcommand\tp{\tilde{p}}
\begin{document}

\title{The Lisle model}
\author{Kevin Pierce}
\maketitle

The Lisle distribution in Laplace space is 
\be \tp(\eta,s) = \frac{1}{s} \frac{k'+s + \theta_1\eta v}{\eta v(s+k_1) + s(s+k')} \ee
It inverts to 
\begin{multline} p(x,t) = \theta_1\delta(x) e^{-k_1t}+\frac{1}{v}e^{-\tau-\xi}\Big\{\theta_1\Big[k_1 \mathcal{I}_0\Big(2\sqrt{\xi\tau}\Big) + k_2\sqrt{\frac{\tau}{\xi}}\mathcal{I}_1\Big(2\sqrt{\xi\tau}\Big)\Big] \\
+ \theta_2\Big[k_1\delta(\tau) + k_2 \mathcal{I}_0\Big(2\sqrt{\xi\tau}\Big) + k_1\sqrt{\frac{\xi}{\tau}}\mathcal{I}_1\Big(2\sqrt{\xi\tau}\Big)\Big] \Big\}.
\end{multline}
Taking derivatives of it implies Laplace moments
\be \frac{\bra \tilde{x}\ket}{v} = \frac{k_1 + \theta_2 s}{s^2(s+k')}\ee 
\be \frac{\bra \tilde{x}^2\ket}{2v^2} = \frac{(s+k_1)(\theta_2 s+k_1)}{s^3(s+k')^2}.\ee 
Inverting the first equation provides the mean 
\be \frac{k'^2 \bra x \ket}{v} = k_1 k' t + (\theta_2 k_2 - \theta_1 k_1)(1-e^{-k't}).\ee
Inverting the second provides the second moment
\begin{multline}
\frac{k'^4}{2v^2}\bra x^2 \ket = \theta_2 k'^2(1-(1+k't)e^{-k't}) \\ + k_1k'(1+\theta_2)((1+e^{-k't})k't-2(1-e^{-k't})) \\+ \frac{k_1^2}{k'^4}(3(1-e^{-k't})-k't(2+e^{-k't})+\frac{(k't)^2}{2}).
\end{multline}
This rearranges to 
\begin{multline}
\frac{k'^4}{2v^2}\bra x^2 \ket 
= \theta_2 \Big[k_1k'(k't-1+e^{-k't}) + k_2^2(1-(1+k't)e^{-k't})\Big] \\
+ k_1k'((1+e^{-k't})k't-2(1-e^{-k't})) \\+ \frac{k_1^2}{k'^4}(3(1-e^{-k't})-k't(2+e^{-k't})+\frac{(k't)^2}{2}).
\end{multline}
This doesn't really simplify, nor does the variance. It only gets worse. Anyway, you can use this to check your results for the more general model.
\bibliography{biblio}
\end{document}



