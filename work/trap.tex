\documentclass[11pt]{article}
% General document formatting
\usepackage[margin=0.75in]{geometry}
\usepackage[parfill]{parskip}
\usepackage[utf8]{inputenc}
\usepackage{subfig}         % side-by-side figures 
% Related to math
\usepackage{amsmath,amssymb,amsfonts,amsthm}
\usepackage{graphicx}
\usepackage{natbib}
\usepackage{titling}
\usepackage{hyperref}
\usepackage{wrapfig}
\usepackage{booktabs} % for wrapping tabulars in accord with
\bibliographystyle{agu}
\setlength{\droptitle}{-5em}   % This is your set screw

%\usepackage[math]{kurier}
\newcommand\be{\begin{equation}} % shortcut to start eq envs 
\newcommand\ee{\end{equation}}   % shortcut to end eq envs
\newcommand\bra{\langle}
\newcommand\ket{\rangle}

\newcommand\om{\omega}
\newcommand\tom{\tilde{\omega}}
\newcommand\tg{\tilde{g}}
\newcommand\tp{\tilde{p}}
\newcommand\tG{\tilde{G}}
\newcommand\El{\mathcal{L}}
\begin{document}

\title{Bedload diffusion through an environment with traps}
\author{Kevin Pierce}
\maketitle

\section{Random walk with trapping}
Consider a tracer which can be either in motion or at rest.
While the tracer is at rest, it can be trapped with a constant probability per unit time. 
If the tracer traps, it is immobile for all time.
The joint distributions that the tracer's sojourn ends, either from the motion state labeled by the index 2 or the rest state labeled by 1T or 1F for "trapped" or "free" are given by
\begin{align}
\om_{1T}(x,t) &= \theta_1\big[1-\Phi(t)\big]g_1(x,t) + \int_0^x dx' \int_0^t dt' \om_2(x',t')\big[1-\Phi(t)\big]g_1(x-x',t-t')\\
\om_{1F}(x,t) &= \theta_1\Phi(t)g_1(x,t) + \int_0^x dx' \int_0^t dt' \om_2(x',t') \Phi(t) g_1(x-x',t-t')\\
\om_2(x,t) &= \theta_2 g_2(x,t) + \int_0^x dx' \int_0^t dt' \om_{1F}(x',t')g_2(x-x',t-t')\\
\end{align}
Taking double transforms gives
\begin{align}
\tom_{1T}(\eta,s) &= \theta_1 \tg_1(\eta,s) + \tom_2(\eta,s)\tg_1(\eta,s)-\tom_{1F}(\eta,s) \\
\tom_{1F}(\eta,s) &= \theta_1\tg_1(\eta,s+\kappa) + \tom_2(\eta,s)\tg_1(\eta,s+\kappa)\\
\tom_2(\eta,s) &= \theta_2 \tg_2(\eta,s) + \tom_{1F}(\eta,s)\tg_2(\eta,s)
\end{align}
This system solves for 
\begin{align}
\tom_{1T}(\eta,s) &= \frac{\theta_1 + \theta_2 \tg_2(\eta,s)}{1-\tg_1(\eta,s+\kappa)\tg_2(\eta,s)}\big\{\tg_1(\eta,s)-\tg_1(\eta,s+\kappa) \big\} \\
\tom_{1F}(\eta,s) &= \frac{\theta_1 + \theta_2 \tg_2(\eta,s)}{1-\tg_1(\eta,s+\kappa)\tg_2(\eta,s)}\tg_1(\eta,s+\kappa)\\
\tom_{2}(\eta,s) &= \frac{\theta_2 + \theta_1 \tg_1(\eta,s+\kappa)}{1-\tg_1(\eta,s+\kappa)\tg_2(\eta,s)}\tg_2(\eta,s)\\
\end{align}
The probabilities of being in state $0$ (trapped), $1$ (rest), and $2$ (motion) are
\begin{align}
p_0(x,t) &= \int_0^t dt' \omega_{1T}(x,t-t')\\
p_1(x,t) &= \theta_1 G_1(x,t) + \int_0^x dx' \int_0^t dt' \omega_2(x',t')G_1(x-x',t-t')\\
p_2(x,t) &= \theta_2 G_2(x,t) + \int_0^x dx' \int_0^t dt' \omega_{1F}(x',t')G_2(x-x',t-t'),
\end{align}
Double transforming gives
\begin{align}
\tp_0(\eta,s) &= \frac{1}{s}\tom_{1T}(\eta,s)\\
\tp_1(\eta,s) &= \theta_1 \tG_1(\eta,s) + \tom_2(\eta,s) \tG_1(\eta,s) \\
\tp_2(\eta,s) &= \theta_2 \tG_2(\eta,s) + \tom_{1F}(\eta,s)\tG_2(\eta,s)\\
\end{align}
The total probability is $p(x,t) = p_0(x,t) + p_1(x,t) + p_2(x,t)$ or 
\begin{multline}
\tp(\eta,s) = \frac{1}{s}\frac{\theta_1 + \theta_2 \tg_2(\eta,s)}{1-\tg_1(\eta,s+\kappa)\tg_2(\eta,s)}\big\{\tg_1(\eta,s)-\tg_1(\eta,s+\kappa) \big\} \\
+\frac{\theta_1\big[\tG_1(\eta,s) + \tg_1(\eta,s+\kappa)\tG_2(\eta,s)\big]+ \theta_2\big[\tG_2(\eta,s) + \tg_2(\eta,s)\tG_1(\eta,s)\big]}{1-\tg_1(\eta,s+\kappa)\tg_2(\eta,s)} \\
\end{multline}
Using the identities $\tg_i(0,s) = \tilde{\psi}_i(s)$ and $\tG_i(0,s) = (1-\tilde{\psi}_i(s))/s,$ it follows that the joint distribution is normalized in space: $\tp(0,s) = \mathcal{L}\{\int_0^\infty dx p(x,t);s\} = 1/s$.
Taking the limit of zero trapping rate $\kappa \rightarrow 0$ the equations reduce to a two-state random walk equivalent to the theories of \citet{Einstein1937} and \citet{Lisle1998}, equation (6.33) in \citet{Weiss1994}. This all seems correct.

\section{Lisle propagators}
Setting $g_1(x,t) = \delta(x)k_1e^{-k_1t}$ and $g_2(x,t) = \delta(x-vt)k_2e^{-k_2 t}$ gives the Laplace representations
\begin{align}
\tg_1(\eta,s) &= \frac{k_1}{k_1 + s}\\
\tg_2(\eta,s) &=  \frac{k_2}{k_2 + \eta v + s} \\
\tG_i(\eta,s) &= \frac{\tg_i(\eta,s)}{k_i}
\end{align}
In terms of these propagators and assuming tracers start in motion $\theta_2 =1$ gives the form
\be \tp(\eta,s) =\frac{1}{s} \frac{(s+\kappa+k')s + \kappa k_2}{\eta v(s + \kappa + k_1) + (s+ \kappa + k')s + \kappa k_2} \ee
with $k' = k_1 + k_2$. Taking $\kappa \rightarrow 0$ gives an earlier result.


\subsection{distribution function}
After a lot of work which is in your notebook, this becomes
\begin{align}
p(x,t) = e^{-(\kappa + k_1)(t-x/v)-k_2x/v}
\Big[&\frac{1}{v}\El_{s\rightarrow t-x/v}^{-1}\Big\{\exp\Big[\frac{k_1k_2}{vs}x\Big]\Big\} \\
&+ \frac{k_2}{v}\El_{s\rightarrow t-x/v}^{-1}\Big\{\frac{1}{s}\exp\Big[\frac{k_1k_2}{vs}x\Big]\Big\} \\
&+ \frac{\kappa k_2}{v}\El_{s\rightarrow t-x/v}^{-1}\Big\{\frac{1}{(s-\kappa-k_1)s}\exp\Big[\frac{k_1k_2}{vs}x\Big]\Big\}\Big]
\end{align}
\begin{align}
p(x,t) = e^{-(\kappa + k_1)(t-x/v)-k_2x/v}
\Bigg[&\frac{1}{v}\delta(t-x/v) + \frac{1}{v}\sqrt{\frac{k_1k_2x}{v(t-x/v)}}\theta(t-x/v)\mathcal{I}_1\Bigg(2\sqrt{\frac{k_1k_2x}{v}\Big(t-\frac{x}{v}\Big)}\Bigg)\\
&+\frac{k_2}{v}\theta(t-x/v)\mathcal{I}_0\Bigg(2\sqrt{\frac{k_1k_2x}{v}\Big(t-\frac{x}{v}\Big)}\Bigg)\\
&+ \kappa\frac{ k_2}{v(\kappa+k_1)}e^{(\kappa+k_1)(t-x/v)}\theta(t-x/v)\mathcal{Q}(x,t)\Bigg]
\end{align}
where the function $\mathcal{Q}$ is
\be \mathcal{Q}(x,t) = \sum_{j=0}^\infty \frac{\big[\frac{k_1k_2x}{v(\kappa+k_1)}\big]^j}{j!j!} \gamma\big(j+1,[\kappa+k_1][t-x/v]\big),\ee
where the $\mathcal{I}_\nu(z)$ are modified Bessel functions of the first kind, and $\gamma(\alpha,z)$ is the lower incomplete gamma function.

\subsection{moments from motion}
\be \partial_\eta \tp(\eta,s) = -v \frac{1}{s}\frac{[(s+\kappa + k')s + \kappa k_2][s+\kappa + k_1]}{[\eta v(s+\kappa +k_1) + (s+ \kappa + k')s+\kappa k_2]^2}\ee
\be \partial_\eta^2 \tp(\eta,s) = 2v^2 \frac{1}{s} \frac{[(s+\kappa + k')s+\kappa k_2][s+\kappa + k_1]^2}{[\eta v(s+\kappa + k_1) + (s+\kappa + k')s+ \kappa k_2]^3}\ee
\be  \frac{\bra\tilde{x}\ket} {v} = \frac{1}{s}\frac{s+\kappa + k_1}{(s+\kappa + k')s + \kappa k_2}\ee
\be \frac{\bra \tilde{x}^2 \ket}{2v^2} = \frac{1}{s} \frac{[s+\kappa + k_1]^2}{[(s+\kappa + k')s + \kappa k_2]^2} \ee
\subsubsection{the mean}
\begin{align}
\frac{\bra x \ket}{v} &= \El^{-1}\Big\{\frac{1}{s}\frac{s+\kappa + k_1}{(s+\kappa + k')s + \kappa k_2};t\Big\} \\
&= \El^{-1}\Big\{\frac{1}{\big[s+\frac{\kappa+k'}{2}\big]^2+\kappa k_2 - \big[\frac{\kappa+k'}{2}\big]^2};t\Big\} + 
\int_0^t du \El^{-1}\Big\{\frac{\kappa + k_1}{\big[s+\frac{\kappa+k'}{2}\big]^2+\kappa k_2 - \big[\frac{\kappa+k'}{2}\big]^2};u\Big\}\\
&= e^{-(\kappa + k')t/2}\El^{-1}\Big\{\frac{1}{s^2 - b^2};t\Big\} + (\kappa + k_1)\int_0^t du e^{-(\kappa + k')u/2}\El^{-1}\Big\{\frac{1}{s^2 - b^2};u\Big\}
\end{align}
here $b^2 = -\kappa k_2 + \big[\frac{\kappa+k'}{2}\big]^2$
Then with Prudnikov 2.1.5.4:
\begin{align}
\frac{\bra x \ket}{v} &= \frac{1}{b}e^{-(\kappa + k')t/2}\sinh(bt) + \frac{(\kappa +k_1)}{b}\int_0^t due^{-(\kappa + k')u/2}\sinh(bu)\\
&=  \frac{1}{b}e^{-at}\sinh(bt) + \frac{\kappa + k_1}{2b}\Big[\frac{1}{b-a}\Big(e^{(b-a)t}-1\Big)+ \frac{1}{a+b}\Big(e^{-(a+b)t}-1\Big)\Big]
\end{align}
where $a=(\kappa + k')/2$ and $b = \sqrt{a^2 -\kappa k_2}$.
The limit of $\kappa \rightarrow 0$ provides
\be \frac{k'^2}{v}\bra x \ket=k_2(1-e^{-k't})+ k_1k't \ee
which aligns perfectly with earlier results.


\subsubsection{the second moment}
A similar approach provides 
\begin{align}
 \frac{\bra x^2 \ket}{2v^2} &= \Big(\frac{d}{dt}\circ + \circ \Big|_{t=0} + 2(\kappa + k_1)\circ + (\kappa+k_1)^2\int_0^t dt \circ \Big)\El^{-1}\Big\{\frac{1}{[(s+a)^2-b^2]^2};t\Big\}\\
&= \Big(\frac{d}{dt}\circ + \circ \Big|_{t=0} + 2(\kappa + k_1)\circ + (\kappa+k_1)^2\int_0^t dt \circ \Big)\El^{-1}\Big\{\frac{1}{[s^2-b^2]^2};t\Big\}e^{-at}\\
&= \Big(\frac{d}{dt}\circ + \circ \Big|_{t=0} + 2(\kappa + k_1)\circ + (\kappa+k_1)^2\int_0^t dt \circ \Big)e^{-at}\frac{1}{2b^3}\Big[bt\cosh(bt)-\sinh(bt)\Big]
\end{align}
using Prudnikov 2.1.5.6.

\bibliography{biblio}
\end{document}



